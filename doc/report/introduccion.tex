\chapter{Introducción}
\label{introduccion}

En este trabajo se explicarán dos artículos de investigación relacionados con la planificación de tareas y movimientos en robots. Más concretamente se trata de los artículos \textit{A Note on ``Solving the Find-Path Problem by Good Representation of Free Space''} y \textit{Quasi-Randomized Path Planning}}.\\

Se comenzará por realizar un pequeño resumen y comentario de cada uno de ellos de forma individual, explicando los algoritmos o ideas que sus autores presentan en ellos. En caso de que sea necesario introducir alguna referencia para la comprensión del artículo asignado, también se presentará un resumen de la refencia a modo de contexto para el lector de este trabajo. Los detalles que van a ser tratados en otras secciones se omitirán de este apartado para evitar redundancias en el trabajo. Una vez presentados ambos artículos, se procederá a realizar una comparativa entre ambos algoritmos, indicando los puntos fuerte y débiles de cada uno.\\

Tras la explicación de los algoritmos presentados en los artículos principal y complementario, se prodecerá a la descripción detallada del caso de uso seleccionado para la implementación del algoritmo. Al ser muy complicada la implementación del algoritmo principal, debido a su naturaleza de crítica a ciertos aspectos concretos de otro artículo, se ha decidido implementar el algoritmo del artículo complementario. Esta sección explica el caso de uso elegido y las razones por las cuales se ha seleccionado este problema como adecuado para el algoritmo a desarrollar.\\

En la siguiente sección se tratarán de forma detallada todos los detalles de la implementación del algoritmo seleccionado, así como los detalles relativos a la simulación. Se describirá el algoritmo implementado paso a paso, haciendo incampié en los aspectos fundamentales que el autor describió en el artículo y que son indispensables para el correcto funcionamiento del algoritmo. También se tratarán las mejoras o cambios introducidos por los alumnos en el algoritmo original, así como el efecto que tiene variar los distintos parámetros del algoritmo sobre el resultado final, tanto en sus resultados y rendimiento como en el tiempo de ejecución del mismo.\\

Por último, se llevará a cabo una conclusión en la que se repasarán los elementos más significativos de este trabajo, a la vez que se reflexionará sobre los resultados obtenidos, enlazándolos y contrastándolos con los resultados presentados en los artículos originales.\\