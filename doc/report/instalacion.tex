\chapter{Instalación y Ejecución del Software}
\label{app:installing}

\section{Instalación del simulador}

Se requiere la instalación del simulador V-REP. Durante este proyecto se ha utilizado la version 3.2.1. Puede descargarse desde la página oficial siguiendo este link: http://www.coppeliarobotics.com/downloads.html

\section{Programación}

El programa se ha realizado en lenguaje Python, por lo que será necesario tener instalado el intérprete de Python. Es este caso se ha utilizado la versión 2.7, que puede descargarse desde: https://www.python.org/downloads/

\section{Dependencias adicionales}

Adicionalmente se requerirá tener instalado:
\begin{itemize}
	\item NumPy: http://www.numpy.org/
	\item OpenCV: http://opencv.org/downloads.html 
	\item cgkit2: http://cgkit.sourceforge.net/download.html
\end{itemize}

\section{Ejecución de la simulación}

Para ejecutar la simulación se deberán seguir los siguientes pasos:
\begin{enumerate}
	\item Abrir V-REP.
	\item Cargar un escenario del proyecto de la carpeta scenes (escenario1.ttt, escenario2.ttt o escenario3.ttt)
	\item Arrancar el simulador pulsando play.
	\item Abrir el archivo platano.py de la carpeta algorithm.
	\item Configurar los parámetros de la simulación. El planificador puede reconfigurarse en la linea 70 del script platano.py. Siguiendo las instrucciones que se encuentran comentadas entre las lineas 48 y 68, pueden variarse los valores que se le pasan a la función para modificar el experimento.
	\item Ejecutar el script platano.py
	\item El script presentará las imágenes correspondietes al plano y pedirá un punto objetivo. Se deberán suministrar las coordenadas del punto final.
	\item Pulsar enter. Si existe una trayectoria para alcanzar el objetivo se mostrará en pantalla.
	\item Si se ha obtenido una trayectoria válida, pulsar de nuevo enter (sobre la ventana path) para comenzar la simulación.
\end{enumerate}
