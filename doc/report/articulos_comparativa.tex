\section{Comparativa}
\label{articulos_comparativa}

Realmente no se puede hacer una comparación directa entre el artículo principal y el complementario, ya que el artículo principal consiste en una crítica al trabajo de R. Brooks y no presenta ningún método de planificación. Sin embargo si que se podria establecer una comparación entre el trabajo de Brooks que trata sobre representar el espacio libre como conos generalizados y el método basado en muestras de datos pseudo-aleatorios.\\

En primer lugar se puede empezar comentando qeu ambos algoritmos presentan una estructura similar  